\documentclass{article}
\usepackage[preprint]{neurips_2019}
% \usepackage[utf8]{inputenc} % allow utf-8 input
% \usepackage[T1]{fontenc}    % use 8-bit T1 fonts
% \usepackage{hyperref}       % hyperlinks
% \usepackage{url}            % simple URL typesetting
% \usepackage{booktabs}       % professional-quality tables
% \usepackage{amsfonts}       % blackboard math symbols
% \usepackage{nicefrac}       % compact symbols for 1/2, etc.
% \usepackage{microtype}      % microtypography
\title{Predicting Stocks Trends Based on News}
\author{
Harsh Dedhiya, Raghav Malhotra, Phumin Walaipatchara\\
Department of Computer Science\\
Boston University\\
\texttt{hdedhiya@bu.edu, raghav20@bu.edu, phuminw@bu.edu}\\
}
\begin{document}
\maketitle
\begin{abstract}
    To add abstract bibibikb
\end{abstract}
\section{Background}
Stocks market has been known to be volatile and sentitive to facctors, including news and statistics.
 Speculation on stocks movement requires complicated techniques and models, still the result is not
 satisfactory.
\subsection{Indicators}
Nothing here for now 

\section{Dataset}
Some text for second section

\section{Na\"ive Bayes}
Describe the result from Na\"ive Bayes approach

\section{Logistic Regression}
Describe the result from logictic regression

\section{Sentiment Analysis}

\section{Recurrent Neural Network}
In order to capture the objective of accurate prediction, Recurrent Neural Network (RNN) is introduced
 because its ability to exhibit internal state (memory). Specifically, Long short-term memory (LSTM),
 a special kind of RNN, is used for implementation as LSTM can deal with vanishing gradient problems
 and is capable of learning long-term dependencies.

\section*{References}
if needed

\end{document}
